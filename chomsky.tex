% Inspired by similar table from Niky @ fituska 2014/2015
% author of this document: Jan Ťulák @ 2015/2016

\documentclass[10pt,a4paper]{article}
\usepackage[czech]{babel}
\usepackage[utf8]{inputenc}
\usepackage{graphicx}
\usepackage{amsmath}
\usepackage{amsfonts}
\usepackage{parskip}
\usepackage{pbox}
\usepackage{datetime}
\usepackage{url}
\newcommand{\tabitem}{~\\~~\llap{\textbullet}~}

\usepackage[a4paper,margin=1cm,landscape]{geometry}
\usepackage{pdflscape}
\pagenumbering{gobble}

\newcommand{\whead}{6cm}
\newcommand{\wreg}{6.5cm}
\newcommand{\wdza}{5.5cm}
\newcommand{\wza}{6.5cm}
\newcommand{\wBothZA}{12cm} % wdza + wza

\newcommand{\wkont}{8cm}
\newcommand{\wrek}{7.5cm}
\newcommand{\wrekv}{9.5cm}
\newcommand{\wrekall}{17cm} % wrek+wrekv

\newcommand{\footer}{\em LaTeX: \url{https://github.com/BroukPytlik/TIN-chomsky}, vytvořeno \today~v~\currenttime}

\begin{document}
%\newgeometry{margin=1cm} % modify this if you need even more space
%\begin{landscape}
\renewcommand{\arraystretch}{1.5}%
\begin{table}
\begin{tabular}{|l|l|l|l|}
	\hline
	& \pbox{\wreg}{\textbf{Regulární jazyky} $\mathcal{L}_3$} &
	\pbox{\wdza}{\vspace{2pt}\textbf{Deterministické bezkontextové jazyky}\vspace{2pt}} &
	\pbox{\wza}{\textbf{Bezkontextové jazyky} $\mathcal{L}_2$}\\
	\hline
	\hline
	\pbox{\whead}{
		\texttt{Automat}\\
		$Q$ -- konečná množina stavů\\
		$\Sigma$ -- vstupní abeceda\\
		$\Gamma$ -- pásková (zásobníková) abeceda\\
		$\delta$ -- přechodová funkce\\
		$q_0 \in Q$ -- počáteční stav\\
		$F\subseteq Q$ -- množina koncových stavů\\
		$Z_0$ -- počáteční symbol na zásobníku
	} & \pbox{\wreg}{
		Konečný -- KA $M = (Q, \Sigma, \delta, q_0, F)$\\
		$\delta: Q\times\Sigma \rightarrow 2^Q$\\
		\\
		Eq. varianty: Deterministický DKA $Q\times\Sigma \rightarrow Q$\\
		Úplný DKA $Q\times\Sigma \rightarrow Q \cup \{trap\}, trap \not \in Q$\\
		Rozšířený DRKA $Q\times (\Sigma\cup \{\varepsilon\}) \rightarrow Q$
	} & \pbox{\wdza}{
		Deterministický -- DZA. Viz ZA, s následujícím omezením.\\
		$\forall q\in Q, a \in \Sigma, z \in \Gamma:$\\
		$|\delta(q,a,z)| \leq 1 \wedge |\delta(q,\varepsilon,z)| = 0 \vee$\\
		$|\delta(q,a,z)| = 0 \wedge |\delta(q,\varepsilon,z)| \leq 1$\\
		\\
		Eq. varianty: Rozšířený DRZA.
	} & \pbox{\wza}{
		\vspace{2pt}
		Zásobníkový -- ZA\\
		$M = (Q, \Sigma, \Gamma, \delta, q_0, Z_0, F)$\\
		$\delta: Q\times(\Sigma\cup\{\varepsilon\})\times\Gamma \rightarrow 2^{Q\times\Gamma^*}$\\
		\\
		Eq. varianty: Rozšířený RZA umí pracovat s libovolným počtem $n$ symbolů na zásobníku najednou. $Q\times(\Sigma\cup\{\varepsilon\})\times\Gamma^* \rightarrow 2^{Q\times\Gamma^*}$
		\vspace{2pt}
	} \\
	\hline
	\pbox{\whead}{
		\vspace{2pt}
		\texttt{Konfigurace}\\
		$q$ -- aktuální stav\\
		$\omega$ -- nezpracovaná část vstupu
		\vspace{2pt}
	} &
	% regulární
	\pbox{\wreg}{$C=(q,\omega) \in Q\times\Sigma^*$}&
	% det. bezk.
	\multicolumn{2}{c|}{\pbox{\wBothZA}{$C=(q,\omega,\gamma) \in Q\times\Sigma^*\times\Gamma^*$}}\\
	\hline
	\pbox{\whead}{\texttt{Relace přechodu}} &
	% regulární
	\pbox{\wreg}{
		$\vdash \subseteq (Q\times\Sigma^*)\times(Q\times\Sigma^*)$\\
		$(q,\omega) \vdash (q',\omega') \Leftrightarrow \omega = a\omega' \wedge q'\in\delta(q,a)$
	} &
	% det. bezk.
	\multicolumn{2}{c|}{\pbox{\wBothZA}{
		\vspace{2pt}
		$\vdash \subseteq (Q\times\Sigma^*\times\Gamma^*)\times(Q\times\Sigma^*\times\Gamma^*)$\\
		$(q,\omega,\gamma)\vdash(q',\omega',\gamma') \Leftrightarrow$
		$\omega = a\omega' \wedge \gamma = Z\alpha \wedge \gamma' = \beta\alpha \wedge (q',\beta)\in\delta(q,a,Z)$
		\vspace{2pt}
	}}\\
	\hline
	\pbox{\whead}{\texttt{Přijímaný jazyk}}&
	% regulární
	\pbox{\wreg}{$L_M=\{\omega|(q_0,\omega)\vdash^*_M(q_f,\varepsilon) \wedge q_f \in F\}$}&
	% det. bezk.
	\pbox{\wdza}{$L_{DZA} \subset L_{ZA}$}&
	% bezk.
	\pbox{\wza}{
		\vspace{2pt}
		$L_{M\varepsilon}=\{\omega|(q_0,\omega,Z_0)\vdash^*_M(q',\varepsilon,\varepsilon) \wedge q' \in Q\}$\\
		$L_{Mf}=\{\omega|(q_0,\omega,Z_0)\vdash^*_M(q_f,\varepsilon,\gamma) \wedge q_f \in F\}$\\
		$L_{Mf\varepsilon}=\{\omega|(q_0,\omega,Z_0)\vdash^*_M(q_f,\varepsilon,\varepsilon) \wedge q_f \in F\}$\\
	}\\
	\hline
	\pbox{\whead}{\texttt{Příklad}}&
	% regulární
	\pbox{\wreg}{
		\vspace{2pt}
		$L=\{a^n|n\geq 0\}$\\
		$S\rightarrow aS|\varepsilon$
		\vspace{2pt}
	}&
	% det. bezk.
	\pbox{\wdza}{
		$L=\{a^nb^n|n\geq 0\}$\\
		$S\rightarrow aSb|\varepsilon$
	}&
	% bezk.
	\pbox{\wza}{
		$L=\{ww^R|\text{R je reverzace řetězce}\}$\\
		$S\rightarrow aSa|bSb|\varepsilon$
	}\\
	\hline
	\pbox{\whead}{\texttt{Tvar pravidel}}&
	% regulární
	\pbox{\wreg}{
		\vspace{2pt}
		\texttt{Pravá lineární gramatika:}\\ $A \rightarrow xB|x; A,B \in N, x \in \Sigma^*$\\
		\texttt{Levá lineární gramatika:}\\$A \rightarrow Bx|x$\\
		\texttt{Regulární gramatika} je taková l. nebo r. gram., kde se poč. neterminál s pravidlem $S\rightarrow\varepsilon$ neobjevuje na pravé straně žádného pravidla, a $x \in \Sigma$ (znak, nikoliv řetězec)
		\vspace{2pt}
	}&
	% det. bezk.
	\multicolumn{2}{c|}{\pbox{\wBothZA}{
		$A \rightarrow \alpha; \alpha \in (N \cup \Sigma)^*$\\
		CNF: $A \rightarrow BC|x$ pro $A,B,C \in N, x \in \Sigma$\\
		GNF: $A \rightarrow x\beta$ pro $A\in N, \beta \in N^*, x \in \Sigma$
	}}\\
	\hline
	\pbox{\whead}{\texttt{Rozhodnutelnost}}&
	% regulární
	\pbox{\wreg}{
		\tabitem\textbf{Prázdnost}: Rozhodnutelný, {\em lze sestrojit DKA, spočítat množinu dostupných stavů a pokud tam není žádný koncový, je jazyk prázdný.}
		\tabitem\textbf{Neprázdnost}: Rozhodnutelný, {\em analogicky jako u prázdnosti.}
		\tabitem\textbf{Náležitost}: Rozhodnutelný, {\em sestrojíme KA a pokud pro daný řetězec přejde do koncového stavu a přijme...}
		\tabitem\textbf{Konečnost}: Rozhodnutelný, {\em viz prázdnost.}
	}&
	% bezk.
	\multicolumn{2}{c|}{\pbox{\wBothZA}{
		\tabitem\textbf{Prázdnost}: Rozhodnutelný, {\em sestrojím gramatiku a spočítám množinu $N_T$ neterminálů generujících terminální řetězce. Pokud $S \in N_T$, tak $L\not = \emptyset$.}
		\tabitem\textbf{Neprázdnost}: Rozhodnutelný, {\em analogicky.}
		\tabitem\textbf{Náležitost}: Rozhodnutelný, {\em udělám průnik KA pro konkrétní řetězec se ZA pro jazyk a ověřím prázdnost.}
		\tabitem\textbf{Konečnost}: Rozhodnutelný, {\em přes pumping lemma.}
	}}\\
	\hline

	\multicolumn{4}{l}{\pbox{20cm}{
	\vspace{10pt}
	\footer
	}}\\
\end{tabular}
\end{table}
\pagebreak

% %%%%%%%%%%%%%%%%%%%%%%%%%%%%%%%%%%%%%%%%%%%%%%%%%%%%%%%%%%%%%%%%%%%%%%%%%%%%%%
% %%%%%%%%%%%%%%%%%%%%%%%%%%%%%%%%%%%%%%%%%%%%%%%%%%%%%%%%%%%%%%%%%%%%%%%%%%%%%%

\renewcommand{\arraystretch}{1.5}%
\begin{table}
\begin{tabular}{|l|l|l|}
	\hline
	\pbox{\wreg}{\textbf{Kontextové jazyky} $\mathcal{L}_1$} &
	\pbox{\wdza}{\textbf{Rekurzivní jazyky}} &
	\pbox{\wza}{\textbf{Rekurzivně vyčíslitelné jazyky} $\mathcal{L}_0$}\\
	\hline
	\hline
	%%%%%%%%%%%%%%%%%%%%%%%%%%%%%%%%%%%%%%%%%%%%%%%%
	% Automat
	% kontextové
	\pbox{\wkont}{
	Lineárně omezený automat - TS, který nikdy neopustí tu část pásky, na které byl zapsaný vstup (např. zarážka vlevo a vpravo od vstupu). Formálně:\\
	$\Sigma \subseteq \Gamma - \{\Delta, <, >\}$
	}&
	% rekurzivní
	\pbox{\wrek}{
	Úplný TS -- TS který zastaví pro každý vstup, např. tak, že má dva koncové stavy.
	}&
	% rek. vyčíslitelné
	\pbox{\wrekv}{
	\vspace{1pt}
	Turingův stroj TS $M=(Q, \Sigma, \Gamma, \delta, q_0, q_F)$,\\
	$\Sigma \subseteq \Gamma - \{\Delta\}, L,R \not \in \Gamma, q_F \in Q$\\
	$\delta: (Q-\{q_F\})\times\Gamma \rightarrow Q\times(\Gamma \cup \{L,R\})$\\
	\\
	Eq. varianty: víc koncových stavů, s oboustranně $\infty$ páskou,\\ nedeterministický $\delta: \cdots \rightarrow 2^{Q\times(\Gamma \cup \{L,R\})}$\\
	k-páskový TS $\delta: (Q-\{q_F\})\times\Gamma_1\times\cdots \rightarrow Q\times(\Gamma_1 \cup \{L,R\})\times\cdots$\\
	\\
	Alternativy se stejnou výpočetní silou:\\
	Automaty s 1 FIFO frontou, vícezásobníkové automaty, automat s čítačem, +-1 operacemi a testem na 0, $\lambda$-kalkul, parciálně rekurzivní funkce
	\vspace{1pt}
	}\\
	\hline
	%%%%%%%%%%%%%%%%%%%%%%%%%%%%%%%%%%%%%%%%%%%%%%%%
	% Konfigurace
	% kontextové
	\pbox{\wkont}{
	\vspace{1pt}
	$C=(q,<\gamma>,n) \in Q\times\Gamma^*\times\mathbb{N}_0, |\gamma|=|\omega| \wedge n\leq|\omega| + 2$\\
	$\gamma$ - obsah pásky, $\omega$ - vstup, $n$ - pozice hlavy
	\vspace{1pt}
	}&
	% rekurzivní
	\multicolumn{2}{c|}{\pbox{\wrekall}{
	$C=(q,\gamma,n)\in Q\times\{\gamma\Delta^\omega\}\times\mathbb{N}_0$\\
	$\gamma$ - obsah pásky, $n$ - pozice hlavy, $\Delta^\omega$ - nekonečná sekvence blanků
	}}\\
	%%%%%%%%%%%%%%%%%%%%%%%%%%%%%%%%%%%%%%%%%%%%%%%%
	% Relace přechodu
	\hline
	% kontextové
	\pbox{\wkont}{
	Stejné jako TS, jen u posunu vpravo podmínka $n\leq|\omega|$ a vlevo $n > 0$.
	}&
	% rekurzivní
	\multicolumn{2}{c|}{\pbox{\wrekall}{
	\vspace{1pt}
	$\vdash \subseteq (Q\times\Gamma^\omega\times\mathbb{N}_0)\times(Q\times\Gamma^\omega\times\mathbb{N}_0)$, $\gamma_n$ - $n$-tý symbol řetězce, $s^n_b(\gamma)$ - řetězec, který vznikne záměnou $n$-tého znaku v $\gamma$ za $b$\\
	$(q,\gamma,n)\vdash(q',\gamma,n+1)$ pro $\delta(q,\gamma_n) = (q',R)$ posun vpravo\\
	$(q,\gamma,n)\vdash(q',\gamma,n-1)$ pro $\delta(q,\gamma_n) = (q',L)$ posun vlevo\\
	$(q,\gamma,n)\vdash(q',s^n_b(\gamma),n)$ pro $\delta(q,\gamma_n) = (q',b)$ změna symbolu na pásce\\
	}}\\
	%%%%%%%%%%%%%%%%%%%%%%%%%%%%%%%%%%%%%%%%%%%%%%%%
	% Jazyk formálně
	\hline
	% kontextové
	\pbox{\wkont}{
	\vspace{1pt}
	$L(M) = \{\omega|(q_0,<\omega>,0)\vdash^*_M(q_F,<\gamma>,n) \wedge \gamma \in \Gamma^* \wedge |\gamma|=|\omega| \wedge n\in\mathbb{N}_0 \wedge n \leq |\omega|+2\}$
	\vspace{1pt}
	}&
	% rekurzivní
	\pbox{\wrek}{
	$L(M) = \{M \text{ rozhoduje jazyk } L(M)\}$
	}&
	% rek. vyčíslitelné
	\pbox{\wrekv}{
	$L(M) = \{\omega|(q_0,\Delta\omega\Delta^\omega,0)\vdash^*_M(q_F,\gamma,n)\wedge \gamma\in\Gamma^* \wedge n\in\mathbb{N}_0\}$
	}\\
	%%%%%%%%%%%%%%%%%%%%%%%%%%%%%%%%%%%%%%%%%%%%%%%%
	% příklad
	\hline
	% kontextové
	\pbox{\wkont}{
	\vspace{1pt}
	$L=\{a^nb^nc^n|n\geq0\}$\\
	$S\rightarrow aSBC|aBC|\varepsilon, CB\rightarrow HB, HB\rightarrow HC,$\\ $HC\rightarrow BC, aB\rightarrow ab, bB\rightarrow bb, bC\rightarrow bc, cC\rightarrow cc$
	\vspace{1pt}
	}&
	% rekurzivní
	\pbox{\wrek}{
	$L=\{(r_1, r_2)|L(r_1) = L(r_2)\wedge r_1, r_2 \in RV\}$\\
	Ekvivalence regulárních výrazů.
	}&
	% rek. vyčíslitelné
	\pbox{\wrekv}{
	$L_{HP} = \{<M>\#<\omega>| M \text{ zastaví pro } \omega\}$\\
	$L_{MP} = \{<M>\#<\omega>| \omega \in L(M)\}$
	}\\
	%%%%%%%%%%%%%%%%%%%%%%%%%%%%%%%%%%%%%%%%%%%%%%%%
	% Tvar pravidel
	\hline
	% kontextové
	\pbox{\wkont}{
	\vspace{1pt}
	$\alpha A\beta \rightarrow \alpha\gamma\beta$, \\
	alternativně $\alpha \rightarrow \beta$, kde $|\alpha|\leq|\beta|$, případně s
	$S \rightarrow \varepsilon$, $S$ není nikdy na pravé straně.\\
	$\alpha,\beta \in (N\cup\Sigma)^*, A \in N$\\
	$\gamma \in (N\cup\Sigma)^+$
	}&
	% rekurzivní
	\multicolumn{2}{c|}{\pbox{\wrekall}{
	$\alpha \rightarrow \beta$\\
	$\alpha \in (N\cup\Sigma)^*N(N\cup\Sigma)^*$\\
	$\beta \in (N\cup\Sigma)^*$
	}}\\
	%%%%%%%%%%%%%%%%%%%%%%%%%%%%%%%%%%%%%%%%%%%%%%%%
	% rozhodnutelnost
	\hline
	% kontextové
	\pbox{\wkont}{
		\tabitem\textbf{Prázdnost}: Není, {\em důkaz přes PCP.}
		\tabitem\textbf{Neprázdnost}: Částečně
		\tabitem\textbf{Náležitost}: Rozhodnutelné, {\em páska je omezena, lze tedy vyčíslit všechny kombinace a zkontrolovat.}
		\tabitem\textbf{Konečnost}: Není, {\em asi zase přes PCP.}
	\vspace{2pt}
	}&
	% rekurzivní
	\pbox{\wrek}{
		\tabitem\textbf{Prázdnost}: Není, {\em důkaz přes PCP.}
		\tabitem\textbf{Neprázdnost}: Částečně
		\tabitem\textbf{Náležitost}: Rozhodnutelné, {\em z definice musí vždy zastavit a říct výsledek.}
		\tabitem\textbf{Konečnost}: Není, {\em důkaz přes PCP.}
	\vspace{2pt}
	}&
	% rek. vyčíslitelné
	\pbox{\wrekv}{
		\tabitem\textbf{Prázdnost}: Není, {\em důkaz přes co-HP, nebo PCP.}
		\tabitem\textbf{Neprázdnost}: Částačně, {\em viz HP.}
		\tabitem\textbf{Náležitost}: Částačně, {\em viz MP.}
		\tabitem\textbf{Konečnost}: Není, {\em viz co-HP.}
	\vspace{2pt}
	}\\
	\hline
	\multicolumn{3}{l}{\pbox{20cm}{
	\vspace{2pt}
	Jazyky mimo $\mathcal{L}_0$ (jazyky algoritmicky neřešitelných problémů):\\
	$co-L_{HP} = \{<M>\#<\omega>| M \text{ nezastaví pro } \omega\}$
	$co-L_{MP} = \{<M>\#<\omega>| \omega \not\in L(M)\}$
	}}\\

\end{tabular}
\end{table}



%\end{landscape}
%\restoregeometry
\end{document}
