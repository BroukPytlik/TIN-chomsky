\documentclass[10pt,a4paper]{article}
\usepackage[czech]{babel}
\usepackage[utf8]{inputenc}
\usepackage{graphicx}
\usepackage{amsmath}
\usepackage{amsfonts}
\usepackage{parskip}
\usepackage{pbox}
\newcommand{\tabitem}{~\\~~\llap{\textbullet}~}

\usepackage[a4paper,margin=1cm,landscape]{geometry}
\usepackage{pdflscape}
\pagenumbering{gobble}

\newcommand{\whead}{6cm}
\newcommand{\wreg}{6.5cm}
\newcommand{\wdza}{5.5cm}
\newcommand{\wza}{6.5cm}
\newcommand{\wBothZA}{12cm} % wdza + wza

\begin{document}
%\newgeometry{margin=1cm} % modify this if you need even more space
%\begin{landscape}
\begin{table}
\begin{tabular}{|l|l|l|l|}
	\hline
	& \pbox{\wreg}{\textbf{Regulární jazyky} $\mathcal{L}_3$} &
	\pbox{\wdza}{\textbf{Deterministické bezkontextové jazyky}} &
	\pbox{\wza}{\textbf{Bezkontextové jazyky} $\mathcal{L}_2$}\\
	\hline
	\hline
	\pbox{\whead}{
		\texttt{automat}\\
		$Q$ -- konečná množina stavů\\
		$\Sigma$ -- vstupní abeceda\\
		$\Gamma$ -- pásková (zásobníková) abeceda\\
		$\delta$ -- přechodová funkce\\
		$q_0 \in Q$ -- počáteční stav\\
		$F\subseteq Q$ -- množina koncových stavů\\
		$Z_0$ -- počáteční symbol na zásobníku
	} & \pbox{\wreg}{
		Konečný -- KA $M = (Q, \Sigma, \delta, q_0, F)$\\
		$\delta: Q\times\Sigma \rightarrow 2^Q$\\
		\\
		Eq. varianty: Deterministický DKA $Q\times\Sigma \rightarrow Q$\\
		Úplný DKA $Q\times\Sigma \rightarrow Q \cup \{trap\}, trap \not \in Q$\\
		Rozšířený DRKA $Q\times (\Sigma\cup \{\varepsilon\}) \rightarrow Q$
	} & \pbox{\wdza}{
		Deterministický -- DZA. Viz ZA, s následujícím omezením.\\
		$\forall q\in Q, a \in \Sigma, z \in \Gamma:$\\
		$|\delta(q,a,z)| \leq 1 \wedge |\delta(q,\varepsilon,z)| = 0 \vee$\\
		$|\delta(q,a,z)| = 0 \wedge |\delta(q,\varepsilon,z)| \leq 1$\\
		\\
		Eq. varianty: Rozšířený DRZA.
	} & \pbox{\wza}{
		Zásobníkový -- ZA\\
		$M = (Q, \Sigma, \Gamma, \delta, q_0, Z_0, F)$\\
		$\delta: Q\times(\Sigma\cup\{\varepsilon\})\times\Gamma \rightarrow 2^{Q\times\Gamma^*}$\\
		\\
		Eq. varianty: Rozšířený RZA umí pracovat s libovolným počtem $n$ symbolů na zásobníku najednou. $Q\times(\Sigma\cup\{\varepsilon\})\times\Gamma^* \rightarrow 2^{Q\times\Gamma^*}$
	} \\
	\hline
	\pbox{\whead}{
		\texttt{konfigurace}\\
		$q$ -- aktuální stav\\
		$\omega$ -- nezpracovaná část vstupu
	} &
	% regulární
	\pbox{\wreg}{$C=(q,\omega) \in Q\times\Sigma^*$}&
	% det. bezk.
	\multicolumn{2}{c|}{\pbox{\wBothZA}{$C=(q,\omega,\gamma) \in Q\times\Sigma^*\times\Gamma^*$}}\\
	\hline
	\pbox{\whead}{\texttt{relace přechodu}} &
	% regulární
	\pbox{\wreg}{
		$\vdash \in (Q\times\Sigma^*)\times(Q\times\Sigma^*)$\\
		$(q,\omega) \vdash (q',\omega') \Leftrightarrow \omega = a\omega' \wedge q'\in\delta(q,a)$
	} &
	% det. bezk.
	\multicolumn{2}{c|}{\pbox{\wBothZA}{
		$\vdash \in (Q\times\Sigma^*\times\Gamma^*)\times(Q\times\Sigma^*\times\Gamma^*)$\\
		$(q,\omega,\gamma)\vdash(q',\omega',\gamma') \Leftrightarrow$
		$\omega = a\omega' \wedge \gamma = Z\alpha \wedge \gamma' = Z\beta \wedge (q',\beta)\in\delta(q,a,\alpha)$
	}}\\
	\hline
	\pbox{\whead}{\texttt{Přijímaný jazyk}}&
	% regulární
	\pbox{\wreg}{$L_M={\omega|(q_0,\omega)\vdash^*_M(q_f,\varepsilon) \wedge q_f \in F}$}&
	% det. bezk.
	\pbox{\wdza}{$L_{DZA} \subset L_{ZA}$}&
	% bezk.
	\pbox{\wza}{
		$L_{M\varepsilon}={\omega|(q_0,\omega,Z_0)\vdash^*_M(q',\varepsilon,\varepsilon) \wedge q' \in Q}$\\
		$L_{Mf}={\omega|(q_0,\omega,Z_0)\vdash^*_M(q_f,\varepsilon,\gamma) \wedge q_f \in F}$\\
		$L_{Mf\varepsilon}={\omega|(q_0,\omega,Z_0)\vdash^*_M(q_f,\varepsilon,\varepsilon) \wedge q_f \in F}$\\
	}\\
	\hline
	\pbox{\whead}{\texttt{Příklad}}&
	% regulární
	\pbox{\wreg}{
		$L=\{a^n|n\geq 0\}$\\
		$S\rightarrow aS|\varepsilon$
	}&
	% det. bezk.
	\pbox{\wdza}{
		$L=\{a^nb^n|n\geq 0\}$\\
		$S\rightarrow aSb|\varepsilon$
	}&
	% bezk.
	\pbox{\wza}{
		$L=\{ww^R|\text{R je reverzace řetězce}\}$\\
		$S\rightarrow aSa|bSb|\varepsilon$
	}\\
	\hline
	\pbox{\whead}{\texttt{Tvar pravidel}}&
	% regulární
	\pbox{\wreg}{
		\texttt{Pravá lineární gramatika:}\\ $A \rightarrow xB|x; A,B \in N, x \in \Sigma^*$\\
		\texttt{Levá lineární gramatika:}\\$A \rightarrow Bx|x$\\
		\texttt{Regulární gramatika} je taková l. nebo r. gram., kde se poč. neterminál $S$ neobjevuje na pravé straně žádného pravidla.
	}&
	% det. bezk.
	\multicolumn{2}{c|}{\pbox{\wBothZA}{
		$A \rightarrow \alpha; \alpha \in (N \cup \Sigma)^*$\\
		CNF: $A \rightarrow BC|x$ pro $A,B,C \in N, x \in \Sigma$\\
		GNF: $A \rightarrow \beta x$ pro $A\in N, \beta \in N^*, x \in \Sigma$
	}}\\
	\hline
	\pbox{\whead}{\texttt{Rozhodnutelnost}}&
	% regulární
	\pbox{\wreg}{
		\tabitem\textbf{Prázdnost}: Rozhodnutelný, {\em lze sestrojit DKA, spočítat množinu dostupných stavů a pokud tam není žádný koncový, je jazyk prázdný.}
		\tabitem\textbf{Neprázdnost}: Rozhodnutelný, {\em analogicky jako u prázdnosti.}
		\tabitem\textbf{Náležitost}: Rozhodnutelný, {\em sestrojíme KA a pokud pro daný řetězec přejde do koncového stavu a přijme...}
		\tabitem\textbf{Konečnost}: Rozhodnutelný, {\em viz prázdnost.}
	}&
	% bezk.
	\multicolumn{2}{c|}{\pbox{\wBothZA}{
		\tabitem\textbf{Prázdnost}: Rozhodnutelný, {\em sestrojím gramatiku a spočítám množinu $N_T$ neterminálů generujících terminální řetězce. Pokud $S \in N_T$, tak $L\not = \emptyset$.}
		\tabitem\textbf{Neprázdnost}: Rozhodnnutelný, {\em analogicky.}
		\tabitem\textbf{Náležitost}: Rozhodnnutelný, {\em udělám průnik KA pro konkrétní řetězec se ZA pro jazyk a ověřím prázdnost.}
		\tabitem\textbf{Konečnost}: Rozhodnutelný, {\em přes pumping lemma.}
	}}\\
	\hline

\end{tabular}
\end{table}
%\end{landscape}
%\restoregeometry
\end{document}
